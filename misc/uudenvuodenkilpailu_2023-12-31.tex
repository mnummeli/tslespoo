\documentclass[a4paper,12pt,twoside]{article}
\usepackage[utf8]{inputenc}
\usepackage[T1]{fontenc}
\usepackage[finnish]{babel}
\usepackage[pdftex,colorlinks]{hyperref}
\begin{document}
\title{Uuden vuoden tietokilpailu}
\author{Anni Niska, Mikko Nummelin}
\date{31.12.2023}
\maketitle
\begin{enumerate}
\item{Mikä on uudenvuodenaatto viroksi?}
  \begin{itemize}
  \item[a)]{uueaastapäev}
  \item[b)]{lihavõttepüha}
  \item[c)]{uuevuoepüha}
  \item[d)]{vanaaastaõhtu}
  \item[e)]{uuenädalalaupäev}
  \end{itemize}
\item{Milloin on seuraavan kerran vuosi, jonka vuosiluku on neljällä jaollinen, mutta joka \emph{ei ole} karkausvuosi?}
  \begin{itemize}
  \item[a)]{alkavana vuonna 2024}
  \item[b)]{vuonna 2026}
  \item[c)]{vuonna 2100}
  \item[d)]{vuonna 2200}
  \item[e)]{vuonna 2400}
  \end{itemize}
\item{Kuka jakaa lahjat venäläisiin lapsiperheisiin joulupukin sijasta?}
  \newline\emph{Kirjoita vastaus.}
\item{Onko suojalasien käyttö ilotulitteita räjäyttäessä Suomessa pakollista?}
  \begin{itemize}
  \item[a)]{ei ole}
  \item[b)]{vain jos räjäyttäjä on alaikäinen}
  \item[c)]{on}
  \end{itemize}
  \newpage
\item{Missä kunnassa sijaitsee Suomen mantereen läntisin piste?}
  \begin{itemize}
  \item[a)]{Kustavissa}
  \item[b)]{Uudessakaupungissa}
  \item[c)]{Vaasassa}
  \item[d)]{Eckerössä}
  \item[e)]{Enontekiöllä}
  \end{itemize}
\item{Kuka Straussin säveltäjäsuvusta sävelsi Tonava kaunoisen?}
  \begin{itemize}
  \item[a)]{Johann Strauss I}
  \item[b)]{Johann Strauss II}
  \item[c)]{Josef Strauss}
  \item[d)]{Eduard Strauss}
  \item[e)]{Maria Strauss}
  \end{itemize}
\item{Mikä seuraavista maista otti eurovaluutan käyttöön ensimmäisenä?}
  \begin{itemize}
  \item[a)]{Liettua}
  \item[b)]{Latvia}
  \item[c)]{Viro}
  \item[d)]{Puola}
  \item[e)]{Kroatia}
  \item[f)]{Tshekki}
  \end{itemize}
\item{Musiikkikappale esitetään C-duurissa. Mikä sävel seuraavista on kappaleessa dominantti?}
  \begin{itemize}
  \item{D}
  \item{E}
  \item{F}
  \item{G}
  \item{A}
  \item{H}
  \end{itemize}
  \newpage
\item{Montako reaalista nollakohtaa on polynomilla $x^3-3x+1$?}
  \begin{itemize}
  \item{0}
  \item{1}
  \item{2}
  \item{3}
  \end{itemize}
\item{Millä seuraavista Kanta-Hämeen kunnista \emph{ei ole} yhteistä rajaa Uudenmaan maakunnan kanssa?}
  \begin{itemize}
  \item[a)]{Riihimäki}
  \item[b)]{Loppi}
  \item[c)]{Tammela}
  \item[d)]{Hausjärvi}
  \item[e)]{Jokioinen}
  \end{itemize}
\item{Mikä seuraavista oli mäkihyppääjä Matti Nykäsen syntymäkaupunki?}
  \begin{itemize}
  \item[a)]{Kuopio}
  \item[b)]{Jyväskylä}
  \item[c)]{Lappeenranta}
  \item[d)]{Helsinki}
  \item[e)]{Lahti}
  \item[f)]{Kouvola}
  \item[g)]{Tampere}
  \end{itemize}
  \newpage
\item{Kuka seuraavista ei ole koskaan soittanut The Rolling Stones:in vakituisessa kokoonpanossa?}
  \begin{itemize}
  \item[a)]{Ian Stewart}
  \item[b)]{Mick Jagger}
  \item[c)]{Keith Richards}
  \item[d)]{Brian Jones}
  \item[e)]{Bill Wyman}
  \item[f)]{Charlie Watts}
  \item[g)]{Mick Taylor}
  \item[h)]{George Harrison}
  \item[i)]{Ronnie Wood}
  \end{itemize}
\item{Laske yhteen kaikki kokonaisluvut yhdestä kahteensataan. Mikä on vastaus?}
  \begin{itemize}
  \item[a)]{20100}
  \item[b)]{10000}
  \item[c)]{15100}
  \item[d)]{17800}
  \item[e)]{31000}
  \end{itemize}
\item{Mikä seuraavista tarkoittaa vaakunoiden tutkimusta?}
  \begin{itemize}
  \item[a)]{heraldiikka}
  \item[b)]{topologia}
  \item[c)]{logistiikka}
  \item[d)]{estetiikka}
  \item[e)]{eksentrismi}
  \item[f)]{rajatiede}
  \end{itemize}
\end{enumerate}
\end{document}

# Oikea rivi

1. d
2. c
3. Pakkasukko
4. c
5. e
6. b
7. c
8. G
9. 3
10. e
11. b
12. h
13. a
14. a
