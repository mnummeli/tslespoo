\documentclass[a4paper,12pt]{amsart}
\usepackage[utf8]{inputenc}
\usepackage[T1]{fontenc}
\usepackage[finnish]{babel}
\usepackage[pdftex,colorlinks]{hyperref}
\usepackage{amsmath}
\usepackage{amssymb}
\begin{document}
\title{Kombinaattorilogiikkaa}
\author{Mikko Nummelin}
\maketitle
\tableofcontents
\newtheorem{esimerkki}{Esimerkki}
\newtheorem{lause}{Lause}
\section{Yleistä}
1930-luvulla matemaatikko Alonzo Church tutki laskettavuuden teoriaa ja keskittyi pohtimaan, millaisia objekteja matemaattiset funktiot oikeastaan ovat ja mitä tyyppiä ne ovat. Jos esimerkiksi $f(n)=n^2,\ n\in\mathbb{N}$, niin $f$ on funktio luonnollisilta luvuilta luonnollisille luvuille $f:\mathbb{N}\to\mathbb{N}$ ja itse $f$ voidaan ymmärtää "objektina" $f=\lambda n.n^2$, jossa $f$ on \emph{tyyppiä} $\mathbb{N}\to\mathbb{N}$. Samoihin aikoihin matemaatikko Haskell Curry (1900-1982) tutki samantapaisia asioita ja keksi esimerkiksi funktioiden argumenttien "ketjuttamisen" sekä Curry-Howard-vastaavuuden eräiden lambda-lauseiden ja todistuvien propositiologiikan aksioomien välillä. Vastaavuus ei ole täydellinen, koska jos lambda-lauseet ovat itseensä viittaavia, niillä ei välttämättä ole vastaavaa propositiologiikan aksioomaa ja jos propositiologiikan aksiooma sisältää negaatioita, sillä ei ole välttämättä vastaavaa lambda-lausetta.

Churchin ja Curryn tutkimukset muodostavat pohjan monille nykyaikaisille ohjelmointikielille ja niiden tavoille käsitellä funktioiden argumentteja ja funktioita arvoina. Eräitä esimerkkejä tällaisista ohjelmointikielistä ovat Javascript, Lisp, Python ja osittain Javan versiot 8:sta eteenpäin. Kurssin tarkoituksena on syventää tietoa funktionaalisen ohjelmoinnin alkuperäisistä matemaattis-loogisista perusteista.
\section{Lambda-laskenta}
Lambda-, eli $\lambda$-laskennan ideana on kuvata funktio objektina, joka on valmis ottamaan vastaan argumentteja ja laskemaan niille arvon. Perinteisessä $\lambda$-laskennassa argumentteja otetaan vastaan vain yksi kerrallaan ja sen seurauksena palautetaan uusi $\lambda$-lause, joka on valmis ottamaan vastaan seuraavan argumentin, kunnes lopullinen arvo ollaan valmis palauttamaan.
\begin{esimerkki}
Tarkastellaan yhteenlaskua $\mathrm{plus}(x,y)=x+y$. Nyt voitaisiin määrittää "ketjuttamalla", että $\mathrm{plus}=\lambda x.\lambda y.x+y,\ x,y\in\mathbb{R}$. Mitä olisi nyt $\mathrm{plus}(x)$? Se olisi $\lambda y.x+y$, eli funktio, joka on vielä valmis ottamaan vastaan uuden argumentin $y$ ja lisäämään sen jo aiemmin annettuun $x$:ään.
\end{esimerkki}
\subsection{Tyypittämätön $\lambda$-laskenta}
Olkoon $v$ mikä tahansa muuttujasymboli ja $E$ lauseke. Tällöin $v$ ja $(\lambda v.E)$ ovat $\lambda$-lausekkeita. Jos $v$ esiintyy lausekkeessa $E$, sanotaan, että tämä $v$:n esiintymä on \emph{sidottu}. Kaikki muuttujien esiintymät, jotka eivät ole sidottuja, ovat \emph{vapaita}. Olkoon $E_1$ ja $E_2$ mitä tahansa lausekkeita. Tällöin $(E_1 E_2)$ on $\lambda$-lauseke, jossa $E_1$ \emph{evaluoi} $E_2$:n. Mitkään muut lausekkeet eivät ole tyypittämättömiä $\lambda$-lausekkeita. \cite{lambdalaskenta} Jos sekaannusta ei tule, $\lambda$-lausekkeissa voidaan jättää ulommat ja vasemmanpuolimmaiset sulkuparit merkitsemättä.
\subsubsection{$\alpha$-konversio}
Lambda-lauseketta saa muuttaa vaihtamalla muuttujasymbolia toiseksi ja päivittämällä sen sidotut esiintymät samannimisiksi. Esimerkiksi $\lambda x.x=\lambda y.y$. Yleisesti $(\lambda.x M[x])\to(\lambda.y M[y])$. Tätä käytetään selventämään tilanteita, joissa muuttujan sidottu esiintymä uhkaa olla päällekkäinen nimeltään toisen muuttujan vapaan esiintymän kanssa.
\subsubsection{$\beta$-reduktio}
Lambda-lauseke evaluoidaan sijoittamalla evaluoitava lauseke evaluoivaan lausekkeeseen argumentin paikalle, eli $((\lambda x.M)E)\to(M[x:=E])$. Esimerkiksi $(\lambda x.\lambda y.x)(\lambda z.z)\to\lambda y.\lambda z.z\to\lambda x.\lambda y.y$. Jälkimmäisessä tehtiin myös $\alpha$-konversio.
\subsubsection{$\eta$-reduktio}
Jos $E$ ei sisällä $v$:n vapaata esiintymää, niin $(\lambda v.E)v\to E$.
\section{SKI-laskenta}
$\lambda$-laskennan säännöt voivat näyttää vähyydestään huolimatta monimutkaisilta ja etenkin pitkissä lausekkeissa oikeanlaisten $\alpha$-konversioiden tekeminen voi olla työlästä. Asian helpottamiseksi on keksitty vaihtoehtoisia merkintätapoja, joista yksi hyvä apuväline on matemaatikko David Turnerin esittelemä SKI-laskenta \cite{skilaskenta}. Kuten huomaamme, on helppoa todistaa, että SKI-laskennalla voidaan ilmaista samat kaavat kuin $\lambda$-laskennallakin ja toisin päin. SKI-laskennan kolme perinteistä kombinaattoria määritellään seuraavasti:
\begin{equation}\label{ski}
\begin{split}
\mathbf{I}x &\to x \\
\mathbf{K}xy &\to x \\
\mathbf{S}xyz &\to xz(yz)
\end{split}
\end{equation}
Jopa näistäkin kolmesta $\mathbf{I}$ on tarkkaan ottaen tarpeeton, koska $\mathbf{SKK}x\to\mathbf{Kx}(\mathbf{Kx})\to x$, eli $\mathbf{SKK}\stackrel{\beta}{=}\mathbf{I}$. Kuten $\lambda$-laskennassa, jos sekaannusta ei tule, SKI-lausekkeissa voidaan jättää ulommat ja vasemmanpuolimmaiset sulkuparit merkitsemättä.
\begin{lause}[Turnerin lause]\label{turner}
Jokaista $\lambda$-lausetta vastaa SKI-lause ja päinvastoin.

{\bf Todistus}: Kaikki $\lambda$-lauseet, jotka ottavat argumentin ovat jotakin seuraavista kolmesta muodosta: (1) $\lambda v.v$, (2) $\lambda v.E$, missä $v$ ei esiinny vapaana $E$:ssä tai (3) $\lambda v.E_1 E_2$, missä $v$ esiintyy vapaana ainakin jommassakummassa $E_1$ tai $E_2$. Nyt $\lambda v.v\stackrel{\beta}{=}\mathbf{I}$, $\lambda v.E\stackrel{\beta}{=}\mathbf{K}E$ ja $\lambda v.E_1 E_2\stackrel{\beta}{=}\mathbf{S}(\lambda v.E_1)(\lambda v.E_2)$. Rekursiivisesti toistamalla päädytään aina SKI-lausekkeisiin ilman sidottuja muuttujia.

Vastaavasti $\mathbf{I}\stackrel{\beta}{=}\lambda v.v$, $\mathbf{K}\stackrel{\beta}{=}\lambda x.\lambda y.x$ ja $\mathbf{S}\stackrel{\beta}{=}\lambda x.\lambda y.\lambda z.xz(yz)$.
\end{lause}
Edellä näimme, että pelkät kombinaattorit $\mathbf{K}$ ja $\mathbf{S}$ muodostavat täydellisen kannan kombinattorilogiikalle. On olemassa jopa kombinaattori $\iota:=\lambda x.\mathbf{SK}$, joka muodostaa täydellisen kannan jopa yksin! \cite{iota} Tämä johtuu siitä, että voidaan todistaa, että $\iota\iota\stackrel{\beta}{=}\mathbf{I}$, $\iota(\iota\iota)\stackrel{\beta}{=}\mathbf{SK}$, $\iota(\iota(\iota\iota))\stackrel{\beta}{=}\mathbf{K}$ ja lopulta $\iota(\iota(\iota(\iota\iota)))\stackrel{\beta}{=}\mathbf{S}$, mutta näin saadut kaavat olisivat hyvin erikoisen näköisiä, pitkiä ja vaikeita käsitellä, joten lähdemme toiseen suuntaan ja ennemmin lisäämme ylimääräisiä johdettuja kombinaattoreita laskennan helpottamiseksi.
\section{SKI-laskennan laajennukset}
Uusien kombinaattorien lisääminen ei varsinaisesti lisää SKI-laskennan laskentakykyä, mutta helpottaa sitä ja mahdollistaa sen havainnollistamisen, kuinka Turnerin lausetta (\ref{turner}) voidaan käytännössä soveltaa johdettaessa uusien kombinaattorien esitys aiemmin mainittujen $\mathbf{S}$, $\mathbf{K}$ ja $\mathbf{I}$ avulla. Otetaan esimerkkinä kombinaattori $\mathbf{F}:=\lambda x.\lambda y.y$:
\begin{equation}
\begin{split}
\lambda x.\lambda y.y &= \\
\lambda x.\mathbf{I} &= \\
\mathbf{KI}.
\end{split}
\end{equation}
On hyvä huomata, että pätee myös $\mathbf{SK}\stackrel{\beta}{=}\mathbf{KI}$, mikä on helposti todistettavissa evaluoimalla kummallakin lauseke $xy$. Otetaan seuraavana esimerkkinä kombinaattori $\mathbf{B}:=\lambda x.\lambda y.\lambda z.x(yz)$, jota sanotaan myös funktionaaliseksi kompositioksi:
\begin{equation}
\begin{split}
\lambda x.\lambda y\lambda z.x(yz) &= \\
\lambda x.\lambda y.\mathbf{S}(\lambda z.x)(\lambda z.yz) &= \\
\lambda x.\lambda y.\mathbf{S}(\mathbf{K}x)y &= \\
\lambda x.\mathbf{S}(\mathbf{K}x) &= \\
\mathbf{S}(\lambda x.\mathbf{S})(\lambda x.\mathbf{K}x) &= \\
\mathbf{S(KS)K}.
\end{split}
\end{equation}
Ennen seuraavien kombinaattorien johtamista on hyvä tehdä huomio, että edellisistä on pääteltävissä, että $\mathbf{B}fg=\lambda v.f(gv)$, mikä merkitsee, että Turnerin lausetta voidaan laajentaa niin, että lausekkeet $\lambda v.E_1 E_2$ voidaan muuntaa muotoon $\mathbf{B}E_1(\lambda v.E_2)$, mikäli $v$ ei esiinny vapaana $E_1$:ssä, mutta esiintyy vapaana $E_2$:ssa. Tämä yksinkertaistaa näiden lausekkeiden käsittelyä, kun $E_1$:tä varten ei jouduta laajentamaan uutta $\lambda$-lauseketta, jos käytettäisiin aiempaa $\mathbf{S}$-kombinaattorin sääntöä lauseessa (\ref{turner}). Otetaan seuraavana esimerkkinä kombinaattori $\mathbf{C}:=\lambda x.\lambda y.\lambda z.xzy$, joka vaihtaa seuraavia argumentteja:
\begin{equation}
\begin{split}
\lambda x.\lambda y\lambda z.xzy &= \\
\lambda x.\lambda y.\mathbf{S}(\lambda z.xz)(\lambda z.y) &= \\
\lambda x.\lambda y.\mathbf{S}x(\mathbf{K}y) &= \\
\lambda x.\mathbf{B}(\mathbf{S}x)\mathbf{K} &= \\
\mathbf{S}(\lambda x.\mathbf{B}(\mathbf{S}x))(\lambda x.\mathbf{K}) &= \\
\mathbf{S(BBS)(KK)}.
\end{split}
\end{equation}
Tämän perusteella voidaan päätellä, että $\mathbf{C}fg=\lambda v.fvg$, eli Turnerin lausetta voidaan edelleen laajentaa niin, että lausekkeet $\lambda v.E_1 E_2$ voidaan muuntaa muotoon $\mathbf{C}(\lambda v.E_1)E_2$, mikäli $v$ esiintyy vapaana $E_1$:ssä, mutta ei esiinny vapaana $E_2$:ssa. Seuraavaksi otetaan kombinaattori $\mathbf{W}:=\lambda x.\lambda y.xyy$, joka kahdentaa jälkimmäisen argumentin:
\begin{equation}
\begin{split}
\lambda x.\lambda y.xyy &= \\
\lambda x.\mathbf{B}x\mathbf{I} &= \\
\mathbf{CSI}.
\end{split}
\end{equation}
Tässä kohden on ylitetty tärkeä raja, jossa uudet kombinaattorilogiikan kannat ovat tulleet mahdollisiksi. $\mathbf{K}$ ja $\mathbf{S}$-kombinaattorit ovat korvattavissa ja myös $\mathbf{B,C,K,W}$ ja $\mathbf{B,C,F,W}$ ovat nyt kantoja, koska on todistettavissa, että $\mathbf{K}\stackrel{\beta}{=}\mathbf{CF}$ ja $\mathbf{S}\stackrel{\beta}{=}\mathbf{B(BW)(BBC)}$. Seuraavassa taulukossa esitetään joustavat säännöt, miten $\lambda$-lausekkeita kannattaa muuttaa nyt esitellyiksi kombinaattoreiksi, jotta esitystapa olisi mahdollisimman ytimekäs:
\begin{table}
\caption{SKI-operaattorien korvaussäännöt}
\begin{tabular}{|c|c|c|}
\hline
Kombinaattorin toiminta & $\lambda\Rightarrow\mathrm{SKI}$ & Milloin käytetään \\
\hline
\hline
$\mathbf{B}xyz\stackrel{\beta}{=} x(yz)$ & $\lambda v.E_1 E_2\stackrel{\beta}{=} \mathbf{B}E_1(\lambda v.E_2)$ & $v\not\in FV(E_1), v\in FV(E_2)$ \\
\hline
$\mathbf{C}xyz\stackrel{\beta}{=} xzy$ & $\lambda v.E_1 E_2\stackrel{\beta}{=} \mathbf{C}(\lambda v.E_1)E_2$ & $v\in FV(E_1), v\not\in FV(E_2)$ \\
\hline
$\mathbf{I}x\stackrel{\beta}{=} x$ & $\lambda v.v\stackrel{\beta}{=} \mathbf{I}$ & identiteettikuvaus \\
\hline
$\mathbf{K}xy\stackrel{\beta}{=} x$ & $\lambda v.E\stackrel{\beta}{=} \mathbf{K}E$ & $v\not\in FV(E)$ \\
\hline
$\mathbf{S}xyz\stackrel{\beta}{=} xz(yz)$ & $\lambda v.E_1 E_2\stackrel{\beta}{=} \mathbf{S}(\lambda v.E_1)(\lambda v.E_2)$ & $v\in FV(E_1), v\in FV(E_2)$ \\
\hline
$\mathbf{W}xy\stackrel{\beta}{=} xyy$ & $\lambda v.E v\stackrel{\beta}{=} \mathbf{W}(\lambda v.E)$ & $v\in FV(E)$ \\
\hline
$\eta$-reduktio & $\lambda v.E v\stackrel{\beta}{=} E$ & $v\not\in FV(E)$ \\
\hline
\end{tabular}
\label{lambdaski}
\end{table}

Kertauksen vuoksi, taulukossa (\ref{lambdaski}) on esitetty laajennetussa muodossa Turnerin lauseen säännöt ja suositeltavin tapa soveltaa niitä, jotta kaavat olisivat lyhyempiä. $\mathbf{B}$:n, $\mathbf{C}$:n ja $\mathbf{W}$:n kaavoissa olisi mahdollista aina käyttää $\mathbf{S}$:n kaavaa, mutta tällöin tarvitaan ylimääräisiä $\mathbf{K}$-operaatioita karsimaan sitä puolta, jossa $v$ ei esiintynyt vapaana ($v\not\in FV(E)$).

Aiemmin mainittiin, että $\mathbf{B,C,K,W}$ on kanta, koska $\mathbf{S}\stackrel{\beta}{=}\mathbf{B(BW)(BBC)}$. Miten tämä kuuluu johtaa? Ideana on, että $\mathbf{B}$ yhdistää kaksi SKI-lausetta, joten jos lauseet $S_1=\lambda x.\lambda y.\lambda z=xyzz$ ja $S_2=\lambda x.\lambda y.\lambda z.\lambda u=xu(yz)$ yhdistää lauseeksi $\mathbf{B}S_1 S_2$, tämä toimii samoin kuin $\mathbf{S}$. Soveltamalla taulukon (\ref{lambdaski}) mukaisia kaavoja, osoittautuu, että $S_1=\mathbf{BW}$ ja $S_2=\mathbf{BBC}$.

\section{Propositiologiikkaa}
Propositiologiikka käsittelee yhden tai useamman totuusarvon välisiä operaatioita. Esimerkiksi seuraavaksi esitetyt operaatiot negaatio $\lnot$ (\ref{negaatio}), konjunktio $\land$ (\ref{konjunktio}), disjunktio $\lor$ (\ref{disjunktio}) ja implikaatio $\to$ (\ref{implikaatio}) on määritelty totuustaulukoiden avulla. Näitä yhdistämällä voidaan määrittää kaikki propositiologiikan lauseet. Taulukossa käytetään symbolia $\bot$ ilmaisemaan epätosia ja symbolia $\top$ tosia lauseita.

% Negaatio
\begin{table}
\caption{Negaatio}
\begin{tabular}{c|c}
$P$ & $\lnot P$ \\
\hline
\hline
$\bot$ & $\top$ \\
\hline
$\top$ & $\bot$
\end{tabular}
\label{negaatio}
\end{table}

% Konjunktio
\begin{table}
\caption{Konjunktio}
\begin{tabular}{c|c|c}
$P$ & $Q$ & $P \land Q$ \\
\hline
\hline
$\bot$ & $\bot$ & $\bot$ \\
\hline
$\top$ & $\bot$ & $\bot$ \\
\hline
$\bot$ & $\top$ & $\bot$ \\
\hline
$\top$ & $\top$ & $\top$ \\
\end{tabular}
\label{konjunktio}
\end{table}

% Disjunktio
\begin{table}
\caption{Disjunktio}
\begin{tabular}{c|c|c}
$P$ & $Q$ & $P \lor Q$ \\
\hline
\hline
$\bot$ & $\bot$ & $\bot$ \\
\hline
$\top$ & $\bot$ & $\top$ \\
\hline
$\bot$ & $\top$ & $\top$ \\
\hline
$\top$ & $\top$ & $\top$ \\
\end{tabular}
\label{disjunktio}
\end{table}

% Implikaatio
\begin{table}
\caption{Implikaatio}
\begin{tabular}{c|c|c}
$P$ & $Q$ & $P \to Q$ \\
\hline
\hline
$\bot$ & $\bot$ & $\top$ \\
\hline
$\top$ & $\bot$ & $\bot$ \\
\hline
$\bot$ & $\top$ & $\top$ \\
\hline
$\top$ & $\top$ & $\top$ \\
\end{tabular}
\label{implikaatio}
\end{table}

On todistettavissa, että pelkästään negaatio ja yksi muista kolmesta esitellystä kahden muuttujan operaattoreista muodostaa täydellisen kannan niin, että kaikki propositiologiikan lauseet on muodostettavissa niistä. Esimerkiksi $P\lor Q=\lnot P\to Q$ ja $P\land Q=\lnot(P\to \lnot Q)$.
\footnote{Kaksi operaattoria, antikonjunktio $\overline\land: \lnot(P\land Q)$ ja antidisjunktio $\overline\lor: \lnot(P\lor Q)$ virittävät yksinäänkin kaikki propositiologiikan lauseet, mutta kaavoista voi tulla pitkiä ja toisteisia.}
Jatkossa konjunktiota ja disjunktiota ei pääasiassa käsitellä.
\subsection{Aksioomat $\mathbf{AK}$, $\mathbf{AS}$, $\mathbf{MT}$ ja $\mathbf{MP}$}
Propositiologiikan lauseet, jotka ovat tosia kaikilla muuttujien arvoilla, ovat \emph{todistuvia} ja lauseet, jotka ovat epätosia kaikilla muuttujien arvoilla, \emph{ristiriitaisia}. $\top$ on yksinkertaisin mahdollinen todistuva lause ja $\bot$ yksinkertaisin mahdollinen ristiriitainen lause. Lauseet, joissa on mahdollista asettaa muuttujat tietyllä tavalla niin, että se olisi tosi, on \emph{tyydytettävä} ja tällainen todeksi tekevä muuttujien sijoitustapa \emph{malli}. Aksiomaattisessa päättelyssä valitaan muutama todistuva lause ja niistä voidaan todistaa muita lauseita päättelysääntöjen avulla.
\begin{lause}[Aksiooma $\mathbf{AK}$ (Implikaation luonti)]\label{ak}
Propositiolause
$$
P\to(Q\to P)
$$
on todistuva.

{\bf Todistus}: Todistetaan totuustaulukon avulla:

\begin{tabular}{c|c|c|c}
$P$ & $Q$ & $Q\to P$ & $P\to (Q\to P)$ \\
\hline
\hline
$\bot$ & $\bot$ & $\top$ & $\top$ \\
\hline
$\top$ & $\bot$ & $\top$ & $\top$ \\
\hline
$\bot$ & $\top$ & $\bot$ & $\top$ \\
\hline
$\top$ & $\top$ & $\top$ & $\top$ \\
\end{tabular}
\end{lause}
\begin{lause}[Aksiooma $\mathbf{AS}$ (Implikaation jakaminen)]\label{as}
Propositiolause
$$
(P\to(Q\to R))\to((P\to Q)\to(P \to R))
$$
on todistuva.

{\bf Todistus}: Todistetaan totuustaulukon avulla:

\begin{tabular}{c|c|c|c|c|c|c|c}
$P$ & $Q$ & $R$ & $Q\to R$ & $P\to(Q\to R)^*$ & $P\to Q$ & $P\to R$ & $(P\to Q)\to(P \to R)^*$ \\
\hline
\hline
$\bot$ & $\bot$ & $\bot$ & $\top$ & $\top$ & $\top$ & $\top$ & $\top$ \\
\hline
$\top$ & $\bot$ & $\bot$ & $\top$ & $\top$ & $\bot$ & $\bot$ & $\top$ \\
\hline
$\bot$ & $\top$ & $\bot$ & $\bot$ & $\top$ & $\top$ & $\top$ & $\top$ \\
\hline
$\top$ & $\top$ & $\bot$ & $\bot$ & $\bot$ & $\top$ & $\bot$ & $\bot$ \\
\hline
$\bot$ & $\bot$ & $\top$ & $\top$ & $\top$ & $\top$ & $\top$ & $\top$ \\
\hline
$\top$ & $\bot$ & $\top$ & $\top$ & $\top$ & $\bot$ & $\top$ & $\top$ \\
\hline
$\bot$ & $\top$ & $\top$ & $\top$ & $\top$ & $\top$ & $\top$ & $\top$ \\
\hline
$\top$ & $\top$ & $\top$ & $\top$ & $\top$ & $\top$ & $\top$ & $\top$ \\
\hline
\end{tabular}
\end{lause}
\begin{lause}[Aksiooma $\mathbf{MT}$ (Modus tollens)]\label{mt}
Propositiolause
$$
(P\to Q)\to(\lnot Q\to\lnot P)
$$
on todistuva.

{\bf Todistus}: Todistetaan totuustaulukon avulla:

\begin{tabular}{c|c|c|c|c}
$P$ & $Q$ & $P\to Q$ & $\lnot Q\to\lnot P$ & $(P\to Q)\to(\lnot Q\to\lnot P)$ \\
\hline
\hline
$\bot$ & $\bot$ & $\top$ & $\top$ & $\top$ \\
\hline
$\top$ & $\bot$ & $\bot$ & $\bot$ & $\top$ \\
\hline
$\bot$ & $\top$ & $\top$ & $\top$ & $\top$ \\
\hline
$\top$ & $\top$ & $\top$ & $\top$ & $\top$ \\
\hline
\end{tabular}
\end{lause}
Tässä yhteydessä on hyvä huomata, että taulukon (\ref{negaatio}) perusteella kaksi negaatiota kumoavat toisensa $\lnot\lnot P=P$ (Negaation eliminaatio).
\begin{lause}[Aksiooma $\mathbf{MP}$ (Modus ponens)]\label{mp}
Propositiolause
$$
(P\land (P\to Q))\to Q
$$
on todistuva.

{\bf Todistus}: Todistetaan totuustaulukon avulla:

\begin{tabular}{c|c|c|c|c}
$P$ & $Q$ & $P\to Q$ & $P\land (P\to Q)$ & $(P\land (P\to Q))\to Q$ \\
\hline
\hline
$\bot$ & $\bot$ & $\top$ & $\bot$ & $\top$ \\
\hline
$\top$ & $\bot$ & $\bot$ & $\bot$ & $\top$ \\
\hline
$\bot$ & $\top$ & $\top$ & $\bot$ & $\top$ \\
\hline
$\top$ & $\top$ & $\top$ & $\top$ & $\top$ \\
\hline
\end{tabular}

Tämä aksiooma toimii myös päättelysääntönä, jolla kolmea muuta aksioomaa voidaan yhdistää tarvittaessa. Sitä merkitään vaihtoehtoisesti myös
$$
P,P\to Q\vdash Q
$$
tai
$$
\frac{P,P\to Q}{\therefore Q}
$$.
\end{lause}

\section{Curry-Howard-vastaavuus}
Edellä on esitetty sekä kombinaattorilogiikan että propositiologiikan perusteita. Näillä on aikanaan havaittu olevan mielenkiintoinen yhteys, jossa kombinaattorilogiikan operaatiot kuvaavat ohjelman ajoa ja propositiologiikan lauseet niihin liittyviä todistuksia. Jos esimerkiksi $n$ on kokonaisluku, sen \emph{tyyppi} on $\mathbb{Z}$, eli $n\in\mathbb{Z}$. Jos taas kyseessä on tilanne, jossa funktio $f$ evaluoi muuttujan $v$ (eli $fv$), $v\in\alpha$ ja $fv\in\beta$, niin $f$:n tyyppi on $\alpha\to\beta$. $f$ on siis tässä tapauksessa kuvaus joukolta $\alpha$ joukolle $\beta$. Merkitään jatkossa yksinkertaisuuden vuoksi $[f]$, jos $f\in[f]$ ja luodaan päättelysääntö, eli jos $\lambda$-lauseessa esiintyy evaluointi $fg$, niin:
$$
[f]=[g]\to[fg].
$$
Sovitaan tässä vaiheessa, että uloimmat ja oikeanpuolimmaiset sulut voidaan jättää merkitsemättä propositiologiikan lauseissa, eli esimerkiksi $(P\to (Q\to R))=P\to Q\to R$. Tarkastellaan seuraavaksi, mikä on kombinaattorin $\mathbf{K}$ tyyppi:
\begin{equation}\label{akjohto}
\begin{split}
\mathbf{K}xy &= \lambda x.\lambda y.x \\
[\lambda x.\lambda y.x] &= [x]\to[y]\to[x] \\
[x] &:= \alpha \\
[y] &:= \beta \\
[\mathbf{K}] &= \alpha\to\beta\to\alpha.
\end{split}
\end{equation}
Havaitaan, että tämä tuottaa samanlaisen lausekkeen kuin aksiooma $\mathbf{AK}$ (\ref{ak}). Entä kombinaattorin $\mathbf{S}$ tyyppi?
\begin{equation}\label{asjohto}
\begin{split}
\mathbf{S}xyz &= \lambda x.\lambda y.\lambda z.xz(yz) \\
[y] &= [z]\to[yz] \\
[xz] &= [yz]\to[xz(yz)] \\
[x] &= [z]\to[xz] = [z]\to[yz]\to[xz(yz)] \\
[z] &:= \alpha \\
[yz] &:= \beta \\
[xz(yz)] &:= \gamma \\
[\mathbf{S}] &= (\alpha\to\beta\to\gamma)\to(\alpha\to\beta)\to\alpha\to\gamma.
\end{split}
\end{equation}
Havaitaan, että tämä on sama kuin aksiooma $\mathbf{AS}$ (\ref{as}). Kombinaattorin $\mathbf{I}$ tyyppi on triviaalisti $\alpha\to\alpha$, mutta koska $\mathbf{I}\stackrel{\beta}{=}\mathbf{SKK}$, niin $\mathbf{I}$:n tyyppi on myös pääteltävissä aksiomaattisesti seuraavasti:
\begin{align}
\mathbf{AK} &:= \alpha\to\beta\to\alpha \label{ai1} \\
\mathbf{AS} &:= (\alpha\to\beta\to\gamma)\to(\alpha\to\beta)\to\alpha\to\gamma \label{ai2} \\
& (\alpha\to\beta)\to\alpha\to\alpha && \text{MP (\ref{ai1},\ref{ai2})} \label{ai3} \\
\mathbf{AI} &= \alpha\to\alpha\qed. && \text{MP (\ref{ai1},\ref{ai3})}
\end{align}
Johdetaan samantapaisesti kombinaattorin $\mathbf{B}$ tyyppi:
\begin{align}
\mathbf{AK} &:= \alpha\to\beta\to\alpha \label{ab1} \\
\mathbf{AS} &:= (\alpha\to\beta\to\gamma)\to(\alpha\to\beta)\to\alpha\to\gamma \label{ab2} \\
& \delta\to(\alpha\to\beta\to\gamma)\to(\alpha\to\beta)\to\alpha\to\gamma && \text{MP (\ref{ab2},\ref{ab1})} \label{ab3} \\
& (\delta\to\alpha\to\beta\to\gamma)\to\delta\to(\alpha\to\beta)\to\alpha\to\gamma && \text{MP (\ref{ab3},\ref{ab2})} \label{ab4} \\
\delta &:= \beta\to\gamma \\
\mathbf{AB} &= (\beta\to\gamma)\to(\alpha\to\beta)\to\alpha\to\gamma\qed. && \text{MP (\ref{ab1},\ref{ab4})}
\end{align}
Samoin kombinaattorin $\mathbf{C}$ tyyppi:
\begin{align}
\mathbf{AK} &:= \alpha\to\beta\to\alpha \label{ac1} \\
\mathbf{AS} &:= (\alpha\to\beta\to\gamma)\to(\alpha\to\beta)\to\alpha\to\gamma \label{ac2} \\
\mathbf{AB} &:= (\beta\to\gamma)\to(\alpha\to\beta)\to\alpha\to\gamma \label{ac3} \\
\beta &:= \beta\to\gamma \\
\gamma &:= (\delta\to\beta)\to\delta\to\gamma \\
& (\alpha\to\beta\to\gamma)\to\alpha\to(\delta\to\beta)\to\delta\to\gamma && \text{MP (\ref{ac3},\ref{ac3})} \label{ac4} \\
\end{align}

\begin{thebibliography}{99}
\bibitem{iota}[Iota] Wikipedia: \href{https://en.wikipedia.org/wiki/Iota\_and\_Jot}{\emph{Iota-laskenta}}
https://en.wikipedia.org/wiki/Iota\_and\_Jot
\bibitem{lambdalaskenta}[Lambda] Wikipedia: \href{https://en.wikipedia.org/wiki/Lambda\_calculus}{\emph{Lambda-laskenta}} https://en.wikipedia.org/wiki/Lambda\_calculus
\bibitem{skilaskenta}[Ski] Wikipedia: \href{https://en.wikipedia.org/wiki/SKI\_combinator\_calculus}{\emph{SKI-laskenta}} https://en.wikipedia.org/wiki/SKI\_combinator\_calculus
\end{thebibliography}
\end{document}
